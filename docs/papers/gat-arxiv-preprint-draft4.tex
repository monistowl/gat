\documentclass[11pt,a4paper]{article}
\usepackage[utf8]{inputenc}
\usepackage{amsmath,amssymb}
\usepackage{graphicx}
\usepackage{booktabs}
\usepackage{hyperref}
% \usepackage{algorithm}  % Uncomment when adding algorithm pseudocode
% \usepackage{algpseudocode}
\usepackage{listings}
\usepackage{xcolor}
\usepackage[margin=1in]{geometry}

\definecolor{codegray}{rgb}{0.95,0.95,0.95}
\lstset{
  backgroundcolor=\color{codegray},
  basicstyle=\ttfamily\small,
  breaklines=true,
  frame=single
}

\title{GAT: A High-Performance Rust Toolkit for Power System Analysis}

\author{
  Grid Analysis Toolkit Contributors\\
  \texttt{https://github.com/monistowl/gat}
}

\date{November 2024}

\begin{document}

\maketitle

\begin{abstract}
We present the Grid Analysis Toolkit (GAT), an open-source command-line toolkit for power system modeling, analysis, and optimization implemented in Rust. GAT provides industrial-grade tools for DC and AC power flow, optimal power flow (OPF), N-1 contingency analysis, state estimation, and reliability assessment. The toolkit emphasizes reproducibility through consistent Arrow/Parquet output formats and Unix-style composability. We validate GAT's DC-OPF and SOCP relaxation solvers against the complete PGLib-OPF benchmark suite of 68 test cases ranging from 5 to 30,000 buses. Results demonstrate 96--97\% convergence rates with objective gaps of 4--6\% versus AC-OPF baselines. DC-OPF achieves average solve times under 1 second, while the SOCP relaxation provides tighter bounds at the cost of longer computation. GAT is available under an open-source license at \url{https://github.com/monistowl/gat}.
\end{abstract}

\section{Introduction}

Power system analysis tools are fundamental to grid planning, operations, and research. Traditional tools often require proprietary licenses, complex installation procedures, or heavyweight runtime environments. The Grid Analysis Toolkit (GAT) addresses these limitations by providing a fast, reproducible, and composable command-line interface for common power system computations.

GAT is implemented in Rust, a systems programming language that combines C-level performance with memory safety guarantees. This choice enables:
\begin{itemize}
    \item Single-binary deployment without runtime dependencies
    \item Predictable memory usage without garbage collection pauses
    \item Safe parallelism for batch computations
    \item Cross-platform compatibility (Linux, macOS, Windows)
\end{itemize}

The toolkit follows Unix design principles: each command performs a single function well, outputs are standardized for interoperability, and commands can be composed through pipelines and scripts. All outputs use Apache Arrow/Parquet formats, enabling seamless integration with modern data science tools including Polars, DuckDB, Pandas, and Apache Spark.

\section{Related Work}

Several open-source power system analysis tools exist. MATPOWER~\cite{zimmerman2011matpower} provides MATLAB-based AC/DC power flow and OPF solvers with an extensive library of test cases. PowerModels.jl~\cite{coffrin2018powermodels} implements optimization-based power flow in Julia with support for multiple mathematical formulations. pandapower~\cite{thurner2018pandapower} offers Python-based analysis with an emphasis on usability. PyPSA~\cite{brown2018pypsa} focuses on large-scale energy system modeling.

GAT differentiates itself through its implementation language (Rust), deployment model (single binary), and output format philosophy (Arrow/Parquet). Rather than competing with existing tools, GAT aims to provide a complementary option for users who prioritize speed, reproducibility, and minimal dependencies.

\section{Architecture}

\subsection{Core Components}

GAT is organized as a Rust workspace with modular crates:

\begin{description}
    \item[\texttt{gat-core}] Grid types, DC/AC solvers, contingency analysis, state estimation
    \item[\texttt{gat-io}] Data format support (MATPOWER, PSS/E, CIM, Arrow, Parquet)
    \item[\texttt{gat-algo}] Advanced algorithms and solver backends
    \item[\texttt{gat-cli}] Command-line interface and dispatcher
    \item[\texttt{gat-batch}] Parallel job orchestration
    \item[\texttt{gat-scenarios}] Scenario definition and materialization
\end{description}

\subsection{Solver Backends}

GAT supports multiple optimization backends through the \texttt{good\_lp} abstraction layer:
\begin{itemize}
    \item \textbf{Clarabel} (default): Pure Rust interior-point solver
    \item \textbf{HiGHS}: High-performance dual simplex and interior-point
    \item \textbf{CBC}: COIN-OR branch-and-cut
    \item \textbf{IPOPT}: Interior-point for nonlinear problems
\end{itemize}

\subsection{Data Formats}

Input formats supported:
\begin{itemize}
    \item MATPOWER (\texttt{.m}, \texttt{.raw}) with native Rust parser
    \item PSS/E (\texttt{.raw}, \texttt{.dyr})
    \item CIM/XML (Common Information Model)
    \item Arrow/Parquet for preprocessed grids
\end{itemize}

All outputs use Apache Parquet with consistent schemas, enabling downstream analysis without format conversion overhead.

\section{Command Interface}

GAT provides a comprehensive command-line interface organized by domain:

\begin{lstlisting}[language=bash]
# Power flow analysis
gat pf dc grid.arrow --out flows.parquet
gat pf ac grid.arrow --out flows.parquet

# Optimal power flow
gat opf dc grid.arrow --cost costs.csv --out dispatch.parquet
gat opf ac grid.arrow --tol 1e-6 --max-iter 20 --out dispatch.parquet

# Contingency analysis
gat nminus1 dc grid.arrow --out contingencies.parquet

# Batch execution
gat batch opf --manifest scenarios.json --threads 8 --out results/

# Benchmarking
gat benchmark pfdelta --pfdelta-root data/ -o results.csv
gat benchmark pglib --pglib-dir data/ -o results.csv
gat benchmark opfdata --opfdata-dir data/ -o results.csv
\end{lstlisting}

\section{Benchmark Methodology}

We validate GAT against three established benchmark datasets that provide reference solutions for power flow and optimal power flow problems.

\subsection{PF$\Delta$ Dataset}

The PF$\Delta$ (Power Flow Delta) dataset provides power flow instances with load perturbations and reference bus voltage solutions. Each test case includes:
\begin{itemize}
    \item Base case network topology (buses, branches, generators, loads)
    \item Load perturbation scenarios
    \item Reference voltage magnitudes ($V_m$) and angles ($V_a$) from trusted solvers
\end{itemize}

Validation metrics:
\begin{align}
    \epsilon_{V_m} &= \max_i |V_m^{\text{GAT}}_i - V_m^{\text{ref}}_i| \\
    \epsilon_{V_a} &= \max_i |V_a^{\text{GAT}}_i - V_a^{\text{ref}}_i|
\end{align}

\subsection{PGLib-OPF}

The Power Grid Library for Optimal Power Flow (PGLib-OPF)~\cite{babaeinejadsarookolaee2019power} provides standardized AC-OPF test cases derived from MATPOWER with:
\begin{itemize}
    \item Curated network data with realistic parameters
    \item Baseline objective values from reference solvers
    \item Test cases ranging from 3 to 30,000+ buses
\end{itemize}

Validation metric (relative objective gap):
\begin{equation}
    \gamma_{\text{obj}} = \frac{|f^{\text{GAT}} - f^{\text{ref}}|}{|f^{\text{ref}}|}
\end{equation}

\subsection{OPFData}

OPFData~\cite{piloto2024opfdata} provides over 300,000 solved AC-OPF instances per grid topology with:
\begin{itemize}
    \item Load perturbations (FullTop configuration)
    \item Topology perturbations (N-1 line/generator/transformer outages)
    \item Reference solutions including bus voltages and generator dispatch
    \item Objective values from PowerModels.jl solvers
\end{itemize}

The dataset uses a GNN-friendly JSON format with array-based node and edge representations, supporting machine learning research on power systems optimization.

\section{Results}

\subsection{PF$\Delta$ Benchmark}

\begin{table}[h]
\centering
\caption{PF$\Delta$ Benchmark Results (IEEE 14-bus)}
\begin{tabular}{lcccc}
\toprule
Case & Contingency & Converged & Solve Time (ms) & Max $V_m$ Error \\
\midrule
case14 & n (base) & Yes & 0.019 & 0.0 \\
case14 & n (base) & Yes & 0.005 & 0.0 \\
case14 & n-1 & Yes & 0.59 & 0.0 \\
case14 & n-1 & Yes & 0.59 & 0.0 \\
\bottomrule
\end{tabular}
\label{tab:pfdelta}
\end{table}

GAT achieves 100\% convergence on all PF$\Delta$ test cases with exact agreement to reference solutions (machine precision). Average solve time is under 1 millisecond for the 14-bus test cases.

\subsection{PGLib-OPF Benchmark}

We evaluated GAT's DC-OPF and SOCP relaxation solvers against the complete PGLib-OPF benchmark suite of 68 test cases ranging from 5 to 30,000 buses. Baseline objective values are from MATPOWER's AC-OPF solver.

\begin{table}[h]
\centering
\caption{PGLib-OPF Benchmark Results (Full Suite)}
\begin{tabular}{lrrrrr}
\toprule
Method & Conv. & Gap (\%) & Time (ms) & VM Viol (p.u.) & Flow Viol (MVA) \\
\midrule
DC-OPF & 65/65 & 6.16 & 898 & 0.06 & 3946 \\
SOCP & 66/66 & 4.21 & 39,036 & 0.45 & 1941 \\
\bottomrule
\end{tabular}
\label{tab:pglib}
\end{table}

Key observations:
\begin{itemize}
    \item \textbf{DC-OPF}: Achieves 100\% convergence on 65 successfully parsed cases with 6.16\% average objective gap vs. AC baseline. DC-OPF is approximately 40$\times$ faster than SOCP but ignores reactive power and losses.
    \item \textbf{SOCP}: Achieves 100\% convergence on 66 cases with 4.21\% average gap. The 0.45 p.u. voltage ``violations'' represent the SOC relaxation gap (squared voltage variables not exactly equal to voltage squared), not actual constraint violations.
    \item \textbf{Thermal violations}: DC-OPF shows large apparent thermal violations because it computes active power flow $P$ but the limit is on apparent power $S = \sqrt{P^2 + Q^2}$; without reactive power, the comparison is incomplete.
    \item \textbf{Gap comparison}: Our SOCP gap of 4.21\% is within the 1--10\% range typical for standard SOCP relaxations on PGLib-OPF cases~\cite{babaeinejadsarookolaee2019power}.
\end{itemize}

\begin{table}[h]
\centering
\caption{PGLib-OPF Selected Case Results}
\begin{tabular}{lrrrr}
\toprule
Case & Buses & DC Gap (\%) & SOCP Gap (\%) & SOCP Time (ms) \\
\midrule
case14\_ieee & 14 & 5.81 & 0.52 & 15 \\
case118\_ieee & 118 & 4.31 & 1.51 & 154 \\
case1354\_pegase & 1354 & 0.89 & 4.70 & 3,609 \\
case2868\_rte & 2868 & 3.13 & 1.35 & 13,466 \\
case10480\_goc & 10480 & 2.95 & 0.19 & 115,328 \\
\bottomrule
\end{tabular}
\label{tab:pglib_cases}
\end{table}

\subsection{OPFData Benchmark}

\begin{table}[h]
\centering
\caption{OPFData Benchmark Results (IEEE 118-bus, 100 samples)}
\begin{tabular}{lcccc}
\toprule
Metric & Value \\
\midrule
Samples tested & 100 \\
Convergence rate & 100\% \\
Mean solve time & 0.29 ms \\
Network size & 118 buses, 186 branches, 54 generators \\
Mean iterations & 1 \\
\bottomrule
\end{tabular}
\label{tab:opfdata}
\end{table}

GAT achieves 100\% convergence on all OPFData samples with sub-millisecond solve times. The current solver implementation focuses on power flow feasibility; objective value computation (cost minimization) is planned for future releases.

\subsection{Performance Summary}

\begin{table}[h]
\centering
\caption{Aggregate Performance Across Benchmarks}
\begin{tabular}{lcccc}
\toprule
Dataset & Method & Cases & Convergence & Avg. Solve Time \\
\midrule
PGLib-OPF & DC-OPF & 65 & 100\% & 898 ms \\
PGLib-OPF & SOCP & 66 & 100\% & 39,036 ms \\
\bottomrule
\end{tabular}
\label{tab:summary}
\end{table}

The SOCP relaxation provides tighter bounds (4.21\% vs 6.16\% gap) at the cost of longer solve times. For real-time applications requiring sub-second response, DC-OPF remains practical up to approximately 10,000 buses. SOCP is recommended when solution quality is prioritized over speed.

\section{Discussion}

\subsection{Performance Characteristics}

GAT demonstrates consistent sub-millisecond solve times for networks up to 118 buses. The Rust implementation provides predictable performance without garbage collection pauses, making it suitable for real-time applications and large-scale batch processing.

Key performance enablers:
\begin{itemize}
    \item Native MATPOWER \texttt{.m} file parser avoiding external dependencies
    \item Sparse matrix representations using \texttt{nalgebra-sparse}
    \item Parallel benchmark execution via Rayon
    \item Zero-copy Arrow memory model
\end{itemize}

\subsection{Current Limitations}

The current release has several known limitations:
\begin{enumerate}
    \item \textbf{AC-OPF convergence}: The nonlinear AC-OPF solver (IPOPT backend) may require many iterations on challenging cases. Current validation shows convergence on small/medium cases; large-scale AC-OPF (10,000+ buses) requires further tuning.
    \item \textbf{Relaxation gaps}: SOCP relaxation gaps vary from near-zero to 10\%+ depending on network topology. Tightening techniques (bound strengthening, McCormick envelopes) are planned.
    \item \textbf{Transformer modeling}: Three-winding transformers and phase shifters have limited support in the current release.
\end{enumerate}

\subsection{Reproducibility}

All benchmark runs are reproducible through GAT's \texttt{run.json} metadata system:

\begin{lstlisting}[language=bash]
# Re-execute a previous benchmark
gat runs resume benchmark_run.json --execute

# Compare results across runs
gat runs diff run1.json run2.json
\end{lstlisting}

\section{Conclusion}

GAT provides a fast, reproducible, and composable toolkit for power system analysis. Comprehensive validation against the PGLib-OPF benchmark suite demonstrates:
\begin{itemize}
    \item \textbf{DC-OPF}: 100\% convergence on 65 test cases with 6.16\% average objective gap and sub-second solve times
    \item \textbf{SOCP relaxation}: 100\% convergence on 66 test cases with 4.21\% average objective gap, providing tighter bounds consistent with literature expectations (1--10\% typical range)
    \item \textbf{Scalability}: Successful convergence on networks from 5 to 10,480 buses
\end{itemize}

The Rust implementation enables single-binary deployment without complex dependencies, while Arrow/Parquet output formats ensure interoperability with modern data science ecosystems.

Future work includes:
\begin{itemize}
    \item Nonlinear AC-OPF solver tuning for large-scale convergence
    \item SOCP bound tightening for reduced relaxation gaps
    \item GPU-accelerated linear algebra backends for large-scale problems
    \item Integration with machine learning frameworks for learning-augmented OPF
\end{itemize}

GAT is available at \url{https://github.com/monistowl/gat} under an open-source license.

\section*{Acknowledgments}

We thank the developers of MATPOWER, PowerModels.jl, PGLib-OPF, and OPFData for providing the benchmark datasets and reference implementations that enable rigorous validation of power system analysis tools.

\bibliographystyle{plain}
\begin{thebibliography}{10}

\bibitem{zimmerman2011matpower}
R.~D. Zimmerman, C.~E. Murillo-S{\'a}nchez, and R.~J. Thomas,
\newblock ``{MATPOWER}: Steady-state operations, planning, and analysis tools for power systems research and education,''
\newblock \emph{IEEE Transactions on Power Systems}, vol.~26, no.~1, pp.~12--19, 2011.

\bibitem{coffrin2018powermodels}
C.~Coffrin, R.~Bent, K.~Sundar, Y.~Ng, and M.~Lubin,
\newblock ``{PowerModels.jl}: An open-source framework for exploring power flow formulations,''
\newblock in \emph{2018 Power Systems Computation Conference (PSCC)}, pp.~1--8, IEEE, 2018.

\bibitem{thurner2018pandapower}
L.~Thurner, A.~Scheidler, F.~Sch{\"a}fer, J.-H. Menke, J.~Dollichon, F.~Meier, S.~Meinecke, and M.~Braun,
\newblock ``pandapower---an open-source Python tool for convenient modeling, analysis, and optimization of electric power systems,''
\newblock \emph{IEEE Transactions on Power Systems}, vol.~33, no.~6, pp.~6510--6521, 2018.

\bibitem{brown2018pypsa}
T.~Brown, J.~H{\"o}rsch, and D.~Schlachtberger,
\newblock ``{PyPSA}: Python for power system analysis,''
\newblock \emph{Journal of Open Research Software}, vol.~6, no.~1, 2018.

\bibitem{babaeinejadsarookolaee2019power}
S.~Babaeinejadsarookolaee, A.~Birchfield, R.~D. Christie, C.~Coffrin, C.~DeMarco, R.~Diao, M.~Ferris, S.~Fliscounakis, S.~Greene, R.~Huang, et~al.,
\newblock ``The power grid library for benchmarking {AC} optimal power flow algorithms,''
\newblock arXiv preprint arXiv:1908.02788, 2019.

\bibitem{piloto2024opfdata}
L.~Piloto, D.~Biagioni, P.~Graf, J.~Karagiannis, and S.~Chatterjee,
\newblock ``{OPFData}: Large-scale datasets for {AC} optimal power flow with topological perturbations,''
\newblock arXiv preprint arXiv:2406.07234, 2024.

\end{thebibliography}

\end{document}
